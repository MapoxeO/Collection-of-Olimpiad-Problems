% shortcuts
\newcommand{\lr}[1]{\left( {#1} \right)}
\newcommand{\floor}[1]{\left \lfloor {#1} \right \rfloor}
\newcommand{\ceil}[1]{\left \lceil {#1} \right \rceil}
\newcommand{\round}[1]{\left \lfloor {#1} \right \rceil}
\newcommand{\rank}[1]{\mathrm{rank}\:#1}
\newcommand{\Eval}[3]{\left.#1\right|_{#2}^{#3}}

\newcommand{\task}[4]{
	\stepcounter{taskcounter}
	\phantomsection \hypertarget{task:\thetaskcounter}{}
	#1
	\subsection*{ \underline{Задача \thetaskcounter.} }
	#2
	\paragraph { \underline{Решение:} }
	#3
	\if\relax\detokenize{#4}\relax%
	\else%
		\paragraph { \underline{Ответ:} } #4%
	\fi%
	\clearpage
}

\newcommand{\newtaskcounter}[1] {
	\newcounter{task#1} \setcounter{task#1}{0}
}

\newcommand{\custompageref}[2] {
	\hyperlink{#1}{#2}
}

\newcommand{\newtheoremwoproof}[1] {
	\stepcounter{\ctheorem}
	\begin{theorem} \hypertarget{theorem:\thectheorem}{}
		#1
	\end{theorem}
}

\newcommand{\newtheoremproof}[2] {
	\stepcounter{ctheorem}
	\phantomsection \hypertarget{theorem:\thectheorem}{}
	\subsection*{ \underline{Теорема \thectheorem.} }
	#1
	\paragraph { \underline{Доказательство:} }
	\  \par
	#2 \newline
	\clearpage
}

\newcommand{\newlemmaproof}[2] {
	\stepcounter{clemma}
	\phantomsection \hypertarget{lemma:\theclemma}{}
	\subsection*{ \underline{Лемма \theclemma.} }
	#1
	\paragraph { \underline{Доказательство:} }
	\  \par
	#2 \newline
}

% \newcommand{\strikeline}[1]{%
%     \tikz[remember picture, baseline=(#1.base)] \node[inner sep=0pt] (#1) {\vphantom{X}};%
% }

\newcommand{\newtaskref}[1] {
	\stepcounter{task#1}
	\hyperlink{category:#1}
	{%
	\ifnum \pdfstrcmp{#1}{text} = 0%
Текстовые задачи%
\else\ifnum \pdfstrcmp{#1}{parameters} = 0%
Задачи с параметрами%
\else\ifnum \pdfstrcmp{#1}{series} = 0%
Ряды, произведения%
\else\ifnum \pdfstrcmp{#1}{diffeqs} = 0%
Дифференциальные уравнения%
\else\ifnum \pdfstrcmp{#1}{integrals} = 0%
Интегралы%
\else\ifnum \pdfstrcmp{#1}{probability} = 0%
Задачи на вероятность%
\else\ifnum \pdfstrcmp{#1}{linal} = 0%
Задачи линейной алгебры%
\else\ifnum \pdfstrcmp{#1}{complex} = 0%
Комплексные числа, ТФКП%
\else ХУИТА \fi\fi\fi\fi\fi\fi\fi\fi
	}%
	\hypertarget{#1:\arabic{task#1}}{}
}