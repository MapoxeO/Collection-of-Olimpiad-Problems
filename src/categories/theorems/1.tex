\newtheoremproof {
    Пусть $A, B, C$ --- матрицы размера $n \times n$ над полем $\mathbb C$, причем $A, B$ --- коммутирующие матрицы, а матрица $C$ выражается как линейная комбинация $A$ и $B$:
    \[
        C = a \cdot A + b \cdot B
    \]
    \par где $a,b \in \mathbb C$.
    \\\\
    Числа $\lambda_k,\;\mu_k$ являются соответственно собственными числами матриц $A$ и $B$ и соответствуют одному собственному вектору.
    Тогда собственные значения матрицы $C$ имеют вид $a \cdot \lambda_k + b \cdot \mu_k, \quad k = \overline{1,n}$
} {
    \par
    Без потери общности рассмотрим собственные векторы матрицы $A$.
    Собственному числу $\lambda_k$ соответствует какой-нибудь собственный вектор $\mathbf x_k$.
    По определению собственных чисел и векторов выходит следующее верное равенство:
    \[
        A \cdot \mathbf x_k = \lambda_k \mathbf x_k
    \]
    \par Умножим равенство на матрицу $B$ слева:
    \[
        B \cdot \lr{A \cdot \mathbf x_k} = B \cdot \lr{\lambda_k \mathbf x_k} =
        \left[A \cdot B = B \cdot A\right] = A \cdot \lr{B \cdot \mathbf x_k} =
        \lambda_k \lr{B \cdot \mathbf x_k}
    \]
    \par Стало быть, вектор $B \cdot \mathbf x_k$ также является собственным вектором матрицы $A$, причём соответствует тому же собственному числу.
    Такое возможно тогда и только тогда, когда векторы $B \cdot \mathbf x_k$ и $\mathbf x_k$ линейно-зависимы:
    \[
        \exists!\;\mu_k\colon B \cdot \mathbf x_k = \mu_k \mathbf x_k
    \]
    \par Это есть определение собственного числа для матрицы $B$ --- вектор $\mathbf x_k$ является собственным для этой матрицы.
    Значит, собственные подпространства матриц $A, B$ совпадают, т.к. эти матрицы коммутируют.
    \par Домножим матрицу $C$ на вектор $\mathbf x_k$:
    \[
        C \cdot \mathbf x_k = \lr{a \cdot A + b \cdot B} \cdot \mathbf x_k = a \cdot A \cdot \mathbf x_k + b \cdot B \cdot \mathbf x_k = \lr{a \cdot \lambda_k + b \cdot \mu_k} \cdot \mathbf x_k
    \]
    Получили определение собственного числа $a \cdot \lambda_k + b \cdot \mu_k$ матрицы $C$, что и требовалось доказать.
}