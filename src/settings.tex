\pagestyle{fancy}
\fancyhf{}
\fancyhead[R]{\href{https://vk.com/mapoxeo}{MapoxeO}}
\fancyfoot[R]{\thepage} 
\fancyhead[L]{Решение олимпиадных и нестандартных задач по математике}
\setcounter{page}{1}
\headsep=10mm 

\usetikzlibrary{patterns}
\usepgfplotslibrary{fillbetween}

\setlength{\footskip}{12.0pt}
\setlength{\headheight}{12.0pt}

\renewcommand{\theequation}{\arabic{equation}}

% Настройка стиля оглавления
\renewcommand{\cftsecfont}{\normalfont\bfseries} % Жирный шрифт для разделов
\renewcommand{\cftsubsecfont}{\normalfont\itshape} % Курсив для подразделов
\renewcommand{\cftsecleader}{\cftdotfill{\cftdotsep}} % Точки между названием и номером страницы

\pgfplotsset{compat=1.18}

% Создаем пользовательский шаблон штриховки
\pgfdeclarepatternformonly{fill_hatch}
{\pgfqpoint{-1pt}{-1pt}} % Начальная точка
{\pgfqpoint{4pt}{4pt}} % Размер шаблона
{\pgfqpoint{4pt}{4pt}} % Смещение
{
	\pgfsetlinewidth{0.4pt} % Толщина линии
	\pgfpathmoveto{\pgfqpoint{0pt}{0pt}}
	\pgfpathlineto{\pgfqpoint{20pt}{20pt}}
	\pgfusepath{stroke}
}

\newtaskcounter{counter}
\newtaskcounter{iter}

\newcounter{dummy} 			\setcounter{dummy}{0}
\newcounter{ctheorem}		\setcounter{ctheorem}{0}
\newcounter{clemma}			\setcounter{clemma}{0}
\newcounter{ccorollary}		\setcounter{ccorollary}{0}

\newtaskcounter{text}
\newtaskcounter{parameters}
\newtaskcounter{series}
\newtaskcounter{diffeqs}
\newtaskcounter{integrals}
\newtaskcounter{probability}
\newtaskcounter{linal}
\newtaskcounter{complex}
