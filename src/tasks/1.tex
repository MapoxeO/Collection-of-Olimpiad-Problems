\task {
    \newtaskref{text}
} {
    Найти, на какую цифру заканчивается десятичная запись числа $\displaystyle \frac{7^{2019}}{5^{2020}}$.
} {
    Хорошо известно, что несократимая дробь вида $\displaystyle \frac ab$ представляется в десятичном виде в виде непериодической дроби тогда и только тогда, когда все делители числа b являются какой-либо степенью делителей числа 10. Так как $5^{2020}$ является степенью делителя числа $10$, тогда наше число $\displaystyle x = \frac{7^{2019}}{5^{2020}}$ представляется в виде конечной десятичной дроби.\\[5mm]
    
    Тогда можно домножить наше число $x$ таким образом на степень $10$, чтобы оно стало натуральным, причем последняя цифра была не нулём. Именно она является последней цифрой десятичной записи нашего числа $x$.\\[5mm]
    
    Домножим $x$ на $10^{2020}$, получим:
    $$ A = x \cdot 10 ^ {2020} = \frac{7 ^ {2019} \cdot 10 ^ {2020}}{5 ^ {2020}} = 2 \cdot 14 ^ {2019} $$
    
    Тогда нам остается узнать, на какую цифру оканчивается число $A$. Поступим следующим образом:
    \begin{equation}
        A \equiv 2  \cdot 14^{2019} \stackrel{^{(*)}}{\equiv} 2 \cdot 4^{2019} \equiv 2 \cdot 2 ^ {4038} \equiv 2^{4039} \pmod{10}
    \end{equation}
    Сравнение $(*)$ справедливо, так как $ 14 \equiv 4 \pmod{10} $.\\[2mm]
    
    Для того, чтобы найти остаток деления $\displaystyle 2^{4039}$, составим таблицу остатков степеней двойки при делении на $10$.\\
    \begin{center} \begin{tabular}{ |c|c|c|c| } 
        \hline
        $k$ & $2^k$ & $2^k \mod 10$ & $i$ \\ \hline
        0 & 1 & 1   & - \\     \hline
        1 & 2 & 2   & 0 \\
        2 & 4 & 4   & 1 \\ 
        3 & 8 & 8   & 2 \\ 
        4 & 16 & 6  & 3 \\    \hline
        5 & 32 & 2  & 0 \\
        6 & 64 & 4  & 1 \\ 
        7 & 128 & 8 & 2\\
        8 & 256 & 6 & 3 \\   \hline
        & \ldots & \ldots &  \ldots \\   \hline
    \end{tabular} \end{center}
    Очевидно, что остатки степеней двойки подчиняются простому периодичному закону.\\
    
    Найдем остаток $i$ деления числа $4039$ на $4$, за вычетом 1, которому служит первая строка с $k = 0$:
        $$ 4039 - 1 \equiv 4038 \equiv 4036 + 2 \equiv 0 + 2 \equiv 2 \pmod{4}$$
    Таким образом, числу $4039$ соответствует остаток $i = 2$, тогда, подставим результат в $(1)$:
        $$ A \equiv 2^{4039} \equiv 2^{3} \equiv 8 \pmod{10}$$
    Стало быть, число $A$ заканчивается на цифру $8$, но это значит, что десятичная запись нашего исходного числа $ \frac{7^{2019}}{5^{2020}}$ заканчивается на ту же цифру.
}{ 8. }
   