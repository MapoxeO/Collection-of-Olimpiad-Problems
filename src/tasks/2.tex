\task {
    \newtaskref{text}
}{
    На ребрах и вершинах куба стоят натуральные числа от $2001$ до $2020$ (все числа различные). Может ли быть, что числа, стоящие на каждом ребре, являются средними арифметическими чисел в соответствующих вершинах?
}{
    Предположим, что утверждение верно. Пусть $X$ — множество чисел, данных в задании. Нетрудно показать, что $ \left| X \right| = 20$. Хорошо известно, что у куба $12$ ребер и $8$ вершин, а сумма их количества в точности равна $12 + 8 = 20$, что равно $\left| X \right|$. Значит, каждому числу из $X$ соответствует единственное ребро или единственная вершина куба. Следовательно, числа $2001$ и $2020$ обязательно будут использоваться для обозначения двух вершин, потому что оба числа не могут быть средним арифметическим двух других различных чисел из $X$. \\
    
    Далее, так как на ребрах и вершинах стоят числа из $X \subset \mathbb{N}$, то любые две смежные вершины должны иметь числа одной четности для того, чтобы их среднее арифметическое было натуральным. Тогда возможно только два случая: все вершины несут только четные числа  или только нечетные числа. Однако, выше мы показали, что числа $2001$ и $2020$ обязательно будут на двух вершинах, а они разной четности, и мы получили противоречие. Следовательно, изначальное утверждение неверно.
}{ Нет, не может. }