\task {
    \newtaskref{integrals}
} {
    Вычислить двойной интеграл $\displaystyle \iint\limits_{D}\abs{\cos{(x+y)}}dS$, где $D\colon\;  0 \leqslant x+y\leqslant \pi \; \land\; xy \geqslant 0$
} { \;\\% для правильного выравнивания
    \begin{wrapfigure}{l}{0.5\textwidth}
        \centering
        \begin{tikzpicture}
            \tikzset{
                hatch distance/.store in=\hatchdistance,
                hatch distance=10pt,
                hatch thickness/.store in=\hatchthickness,
                hatch thickness=2pt
            }
            \makeatletter
            \pgfdeclarepatternformonly[\hatchdistance,\hatchthickness]{flexible hatch}
            {\pgfqpoint{0pt}{0pt}}
            {\pgfqpoint{\hatchdistance}{\hatchdistance}}
            {\pgfpoint{\hatchdistance-1pt}{\hatchdistance-1pt}}%
            {
                \pgfsetcolor{\tikz@pattern@color}
                \pgfsetlinewidth{\hatchthickness}
                \pgfpathmoveto{\pgfqpoint{0pt}{0pt}}
                \pgfpathlineto{\pgfqpoint{\hatchdistance}{\hatchdistance}}
                \pgfusepath{stroke}
            }
            \makeatother
            \begin{axis} [
                axis equal,
                axis x line=center,
                axis y line=center,
                xlabel={$x$},
                ylabel={$y$},
                xtick={pi},
                ytick={pi},
                xticklabels={$\pi$},
                yticklabels={$\pi$},
                xlabel style={below right},
                ylabel style={above left},
                xmin=-0.25,
                xmax=3.5,
                ymin=-0.25,
                ymax=3.5
            ]
                \addplot[name path = line, domain=0:pi, thick] { pi-x };
                \addplot[name path = down, domain=0:pi, smooth, thick] {0};
                \addplot [pattern=fill_hatch, pattern color=gray] fill between [of=down and line, soft clip={domain=-1:pi}];
                \addplot[smooth, thick] coordinates { (0, 0) (0, pi)};
                
                \node at (axis cs: 3, 1.5) {$D$};
                \node at (axis cs: -0.1, -0.15) {$0$};
                \draw [->, thick, black] (axis cs: 2.9, 1.52) to[bend right=50] (axis cs: 1.2, 1.2);
            \end{axis}
        \end{tikzpicture}
        
        \caption*{Область \(D\) в прямоугольной системе координат. }
    \end{wrapfigure}

    Наша область $D$, лежащая в I четверти координатной плоскости, представляет собой прямоугольный треугольник.

    Введём замену координат
    \begin{equation*}
        \begin{cases}
            x + y = u \\
            y = v x
        \end{cases}
    \end{equation*}

    Выражая \(x, y\) из этой системы, получим
    \begin{equation*}
        \begin{cases}
            x = \frac{u}{1+v} \\
            y = \frac{uv}{1+v}
        \end{cases}
    \end{equation*}

    В таких координатах область \(D\) будет выражаться следующим образом:
    \[
        D^{uv} = \left\{(u,v) \in \mathbb{R}^2 \colon 0 \leqslant u \leqslant \pi \land v \geqslant 0\right\}
    \]
    
    \vspace{4mm}

    \WFclear

    \par Для того, чтобы перейти к новым координатам в наших интегралах, необходимо вычислить Якобиан \(J\) преобразования координат:
    \[
        J = \frac{\partial\lr{x, y}}{\partial\lr{u, v}} =
        \left|\begin{matrix}
            x'_u & y'_u \\
            x'_v & y'_v \\
        \end{matrix}\right| = \frac{1}{\lr{1 + v}^3}
        \left|\begin{matrix}
            1 & v \\
            -u & u \\
        \end{matrix}\right| =
        \frac{-u\lr{v+1}}{\lr{1 + v}^3} = \frac{-u}{\lr{1 + v}^2}
    \]

    Теперь можно\footnotemark[1] перейти к координатам \((u, v)\) в интеграле:
    \[
        \iint\limits_{D} \abs{\cos\lr{x+y}} dS = \iint\limits_{D^{uv}} \abs{\cos\lr{u}} \cdot |J|dudv = \iint\limits_{D^{uv}} \abs{\frac{-u\cos u}{\lr{1 + v}^2}} dudv = \left[u \geqslant 0\right] = \iint\limits_{D^{uv}} \frac{u \abs{\cos u}}{\lr{1 + v}^2} dudv
    \]

    \footnotetext{
        Рассуждая более строго, достаточным условием перехода к новым координатам является \(\forall (u, v) \in D^{uv}~\Rightarrow J(u, v)~\neq~0\). В нашем же случае, при \(u = 0\), оно не соблюдается.
        Для разрешения этой ситуации предлагается взять число \(\varepsilon > 0 \) и рассмотреть интеграл по области \(D_\varepsilon^{uv} = \left\{(u, v) \in \mathbb{R}^2 \colon \varepsilon \leqslant u \leqslant \pi \land v \geqslant 0\right\}\).
        Затем вычислить предел интеграла при \(\varepsilon \to 0\).
    }

    Раскроем этот интеграл как повторный и решим его, не забывая про модуль при косинусе
    \begin{align*}
        &\iint\limits_{D^{uv}} \frac{u \abs{\cos u}}{\lr{1 + v}^2} dudv = \int\limits_{0}^{\pi} du \int\limits_{0}^{+\infty} \frac{u \abs{\cos u}}{\lr{1 + v}^2} dv = \int\limits_{0}^{\pi} u \abs{\cos u} du \cdot \int\limits_{0}^{+\infty} \frac{dv}{\lr{1 + v}^2} = \\
        &= \lr{\int\limits_{0}^{\frac{\pi}2} u \cos u du - \int\limits_{\frac{\pi}2}^{\pi} u \cos u du} \cdot \int\limits_{0}^{+\infty} \frac{d\lr{1 + v}}{\lr{1 + v}^2} =
        \begin{bmatrix}
            & D & I \\
            + & u & \cos u \\
            - & 1 & \sin u \\
            + & 0 & -\cos u
        \end{bmatrix} = \\
        &= \lr{\Eval{\lr{u \sin u - \cos u}}{0}{\frac{\pi}2} - \Eval{\lr{u \sin u - \cos u}}{\frac{\pi}2}{\pi}} \cdot \lr{\Eval{\frac1{1+v}}{+\infty}{0}} = \\
        &= \lr{\lr{\frac\pi 2 \cdot 1 - 0} - \lr{0 - 1} - \lr{\pi \cdot 0 - (-1)} + \lr{\frac\pi 2 \cdot 1 - 0}} \cdot \lr{\frac{1}{1 + 0} - 0} = \pi
    \end{align*}


    \WFclear
} {
    $\pi$
}