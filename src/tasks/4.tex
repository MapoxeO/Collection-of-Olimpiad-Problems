\task {
    \newtaskref{series}
} {
    Найти сумму ряда
    $$\sum_{n=1}^{\infty}{\frac{n+1}{3^n}}$$
} {
    Обозначим сумму как $S$ и преобразуем ее следующим образом:
    $$S = \sum_{n=1}^{\infty}{\frac{n+1}{3^n}} = \left[
    \begin{matrix}
        n + 1 = p \Rightarrow n = p - 1\\
        n = 1 \Rightarrow p = 2\\
        n \rightarrow \infty \Rightarrow p \rightarrow \infty \\
    \end{matrix} \right] =
    \sum_{p=2}^{\infty}{\frac{p}{3^{p - 1}}} \stackrel{^{(*)}}{=} 3 \cdot \lr{\sum_{p=1}^{\infty}{\frac{p}{3^{p}}} - \frac13} =
    3 \cdot \underbrace{\sum_{n=1}^{\infty}{\frac{n}{3^n}}}_\text{K} - 1$$
    Последнее равенство справделиво, так как не важно, как называется переменная суммирования, поэтому можно просто переобозначить ее снова как $n$. Равенство $(*)$ верное, так как можно вынести множитель из суммы и добавить дополнительный член суммы. Обозначим сумму в последнем равенстве как $K$. Выразим ее через $S$:
    
    $$ S = 3K - 1 \Rightarrow K = \frac{S + 1}3 $$
    Еще немного преобразуем сумму $S$:
    $$S = \sum_{n=1}^{\infty}{\frac{n+1}{3^n}} = \underbrace{\sum_{n=1}^{\infty}{\frac{n}{3^n}}}_\text{K} + \sum_{n=1}^{\infty}{\frac1{3^n}} \eqno(1)$$
    
    Подставим в $(1)$ выражение для $K$ и выразим $S$:
        $$ S = \frac{S+1}3 + \sum_{n=1}^{\infty}{\frac1{3^n}} \quad \Rightarrow \quad S = \frac12 + \frac32 \sum_{n=1}^{\infty}{\frac1{3^n}} \eqno(2)$$
    
    Остается найти последнюю сумму. Хорошо известно, что это разложение функции $\displaystyle \frac{x}{1-x}$ в ряд. Воспользуемся этим фактом, чтобы вычислить эту сумму:
        $$ \frac{x}{1-x} = \sum_{n = 1}^{\infty}{x^n} \quad \Rightarrow \quad \sum_{n=1}^{\infty}{\frac1{3^n}} = \frac{\frac13}{1 - \frac13} = \frac1{3-1}=\frac12$$
        
    Подставим получившееся значение в $(2)$, и мы получим окончательный ответ:
        $$ S = \frac12 + \frac32 \cdot \frac12 = \frac54 = 1.25 $$
}{ 1.25 }