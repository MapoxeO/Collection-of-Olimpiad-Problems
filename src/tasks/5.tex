\task {
    \newtaskref{text} \newtaskref{diffeqs}
}{
    За $10$ минут чай охладился от $100^\circ\text{C}$ до  $60^\circ\text{C}$. За какое время он остынет до $25^\circ\text{C}$, если температура воздуха в комнате $20^\circ\text{C}$, а скорость остывания пропорциональная разности температур чая и среды?
}{
    Пусть $T(t)$ --- функция температуры чая от времени в минутах и $T_0 = 20$ --- температура окружающей среды \textit{(воздуха)}. Тогда условие скорости остывания можно записать в виде:
        $$ \frac{\mathrm{d}T}{\mathrm{d}t} = k(T - T_0) \  \text{---  Д.У. $1^\text{\underline{го}}$ порядка с разделяющимися переменными, где $k$ --- какая-то постоянная.}$$
        
    Решим его, разделив переменные:
        $$\frac{\mathrm{d}T}{T-T_0}=k\mathrm{d}t$$
        $$\int \frac{\mathrm{d}T}{T-T_0} = \int k\mathrm{d}t \quad  \Leftrightarrow \quad \ln \left| T-T_0\right| = kt + C_1 \quad  \Leftrightarrow \quad T(t) = T_0 + Ce^{kt}$$
    
    Решим задачу Коши. Подставим начальные условия $T_0 = 20,\ T(0) = 100,\ T(10) = 60$ в получившиееся уравнение. \\[6mm]
    Получим:
    $$ \begin{cases}
        100 = 20 + Ce^{k \cdot 0} \\
        60 = 20 + Ce^{k \cdot 10}
    \end{cases} \Leftrightarrow 
    \begin{cases}
        80 = C \\
        40 = Ce^{k \cdot 10}
    \end{cases} \Leftrightarrow 
    \begin{cases}
        C = 80\\
        e^k = 2^{-0.1}
    \end{cases}$$
    
    Тогда наше уравнение зависимости температуры от времени примет вид:
        $$T(t) = 20 + 80\cdot2^{-0.1t}$$
        
    Вычислим искомое значение времени, за которое чай остынет до температуры $20^{\circ}\text{C}$:
        $$ 25 = 20 + 80 \cdot 2^{-0.1t}$$
        $$ \frac1{16} = 2^{-0.1t} \  \Rightarrow \  2^{-4} = 2^{-0.1t} \ \Rightarrow \  t = 40\  \text{мин.}$$
}{ 40 мин. }