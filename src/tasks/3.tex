\task {
    \newtaskref{series}
} {
    Доказать неравенство $\displaystyle \frac1{2020} < \ln{\frac{2020}{2019}} < \frac1{2019}$. \\[2mm] \textit{Подсказка: Использовать разложение функции $y = \ln{(1+x)}$ в ряд Тейлора}
} {
     Выведем выражение ряда Маклорена \textit{(частный случай ряда Тейлора при $x_0 = 0$)} для функции $y = \ln{(1+x)}$. Для этого необходимо найти общую формулу производной $n\text{-го}$ порядка в точке $x_0 = 0$. Найдем несколько первых производных этой функции:
    \begin{align*}
        y &= \ln(1+x) \\
        y' &= (1+x)^{-1} \\
        y'' &= (-1)\cdot(1+x)^{-2} \\
        y''' &= (-1)\cdot(-2)\cdot(1+x)^{-3} \\
        y^{(4)} &= (-1)\cdot(-2)\cdot(-3)(1+x)^{-4} \\
        &\ldots \\
        y^{(n)} &= (-1)^{n-1}(n-1)!\:(1+x)^{-n}
    \end{align*}
    Последняя строчка представляет собой предположение формулы для $n \in \mathbb{N}$. Докажем эту формулу методом мат. индукции: \\

    База индукции: $\displaystyle y^{(1)} = (-1)^{1 - 1}(1 - 1)!\:(1+x)^{-1} = (1+x)^{-1} = y'$ \\
    \indent Шаг индукции: $\displaystyle y^{(n+1)} = (-1)^nn!\:(1+x)^{-n-1}$\\
    \indent По формуле вычисления производной высших порядков: \\
    \indent $\displaystyle y^{(n+1)} = \lr{y^{(n)}}'=(-1)^{n-1}(n-1)!\:(-n)(1+x)^{-n-1} = (-1)^{n}n!\:(1+x)^{-n-1}$\\[2mm]
            
    Стало быть, шаг индукции доказан. Следовательно, формула справедлива для $\forall n \in \mathbb{N}$ по методу мат. индукции.\\
    Вычислим $\displaystyle y^{(n)}(0) = (-1)^{n-1}(n-1)!$ а также $y^{(0)}(0) = y(0) = \ln{1} = 0$.\\
    Тогда ряд Маклорена для нашей функции примет вид:
        $\displaystyle y = \ln{(1+x)} = \sum_{n=0}^{\infty}{ \frac{y^{(n)}(0)}{n!}x^n } = \sum_{n=1}^{\infty}{ \frac{(-1)^{n-1}}{n}x^n }$\\
    Так оценим число $\displaystyle \ln\frac{2020}{2019}$ сверху:
        $$ \ln{\frac{2020}{2019}} =  \frac1{2019}-\frac1{2\cdot2019^2} + \ldots$$
    Ряд знакочередующийся, и каждый последующий член меньше по модулю предыдущего согласно теореме Лейбница, следовательно справедливо равенство:
    \begin{equation*}
        \ln\frac{2020}{2019} < \frac1{2019} \eqno(1)
    \end{equation*}\\
    Оценим выражение снизу, для этого оценим аргумент логарифма снизу. Перепишем его в виде:
        $$\displaystyle \frac{2020}{2019} = 1 + \frac1{2019} > 1 + \frac1{1!\cdot2020} + \frac1{2!\cdot2020^2} + \frac1{3!\cdot2020^3} + \ldots = e^{\frac1{2020}}$$
    Правая часть этого неравенства представляет собой ряд Маклорена функции $e^x$ в точке $x = \frac1{2020}$. Прологарифмируем его и получим оценку снизу:
        $$ e^{\frac1{2020}} < \frac{2020}{2019}\  \Leftrightarrow \ \frac1{2020} < \ln\frac{2020}{2019} \eqno(2)$$\\
    Из неравенств (1) и (2) получаем окончательную оценку: $ \frac1{2020} < \ln{\frac{2020}{2019}} < \frac1{2019}$, что и требовалось доказать.
}{}