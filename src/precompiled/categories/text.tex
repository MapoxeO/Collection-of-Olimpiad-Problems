\hypertarget{category:text}{}
\section{Текстовые задачи}
\hyperlink{text:1} {\large{\textbf{ Задача 1(1)}}}
Найти, на какую цифру заканчивается десятичная запись числа $\displaystyle \frac{7^{2019}}{5^{2020}}$.\\[4mm]
\hyperlink{text:2} {\large{\textbf{ Задача 2(2)}}}
На ребрах и вершинах куба стоят натуральные числа от $2001$ до $2020$ (все числа различные). Может ли быть, что числа, стоящие на каждом ребре, являются средними арифметическими чисел в соответствующих вершинах?\\[4mm]
\hyperlink{text:3} {\large{\textbf{ Задача 3(5)}}}
За $10$ минут чай охладился от $100^\circ\text{C}$ до  $60^\circ\text{C}$. За какое время он остынет до $25^\circ\text{C}$, если температура воздуха в комнате $20^\circ\text{C}$, а скорость остывания пропорциональная разности температур чая и среды?\\[4mm]
\clearpage