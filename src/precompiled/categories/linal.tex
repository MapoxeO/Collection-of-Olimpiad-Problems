\hypertarget{category:linal}{}
\section{Задачи линейной алгебры}
\hyperlink{linal:1} {\large{\textbf{ Задача 1(16)}}}
Найти максимальное значение действительного параметра $a$, зависящее от натурального числа $n$,
	что для линейного оператора $\mathcal F\colon \mathcal L^n \to \mathcal L^n$,
	заданного матрицей $\mathcal A_{\mathcal F}$ в тривиальном базисе этого пространства и для любого вектора $\forall\mathbf{x} \in \mathcal L^n(\mathbb R)$,
	будет верно неравенство:
	$$ \mathbf x ^T \cdot \mathbf x \geqslant \mathbf x^T \cdot \mathcal F(a\cdot\mathbf x)$$
	
	\par где $\mathcal L^n(\mathbb R)$ -- $n$-мерное линейное пространство, 
	и $\mathcal A_{\mathcal F} = \left\{\delta_{i+1,\;j}\right\}$, 
	и $\delta_{i,\,j}=
	\begin{cases}
		1, &\text{если }i=j\\
		0, &\text{иначе}
	\end{cases}$\\[4mm]
\hyperlink{linal:2} {\large{\textbf{ Задача 2(19)}}}
Найти произведение
    $$
        \prod_{a \in A} \begin{bmatrix}
            \frac{1}{a^2} & 1 \\
            1 & \frac{1}{a^2}
        \end{bmatrix}
    $$
    \par в двух случаях:
    \begin{enumerate}
        \item $A=\mathbb{P} - \mbox{множество всех простых чисел}$;
        \item $A=\mathbb{N} - \mbox{множество натуральных чисел}$.
    \end{enumerate}\\[4mm]
\clearpage